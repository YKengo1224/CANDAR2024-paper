\section{Conclusion}\label{section:conclusion}

%% 本論文では三つの実験を行い、眞邊らが開発したリアルタイム振動抑制システムの
%% 抽出精度と振戦抑制の評価を行った。
%% 検証プラットフォームを用いた振動抽出実験では、
%% オプティカルフローの探索ウインドウサイズやセルサイズを大きくすることが
%% 振動の抽出精度の向上に効果的であることがわかった。
%% また、BMFLCの重みの収束速度を調整するゲインパラメータは、
%% 今回の評価においては、$2^{-7}$,$2^{-8}$,$2^{-9}$が適切であることが分かった。

%% 逆位相波加振装置を用いた加振実験では、意図しない周波数成分が逆位相波加振装置
%% の加振振動に含まれることがわかった。
%% 最後に行った制振実験では,振戦振動の周波数成分の低下がみられたが,
%% 意図しない周波数成分の影響などにより,
%% その他の周波数成分の振動成分が増加が認められた.

%% 今後の課題として,逆位相波発生装置において,意図しない周波数成分が発生する原因の解明や,非定常振動における振戦の制振評価,
%% 振動の変位量を高精度で推定可能なアルゴリズムの検討を行う.


In this paper, three experiments were conducted to evaluate the extraction accuracy
and tremor suppression of the real-time vibration extraction system of Manabe et al.

Vibration extraction experiment
Vibration extraction experiments using the validation platform showed that
increasing the optical flow search window size and cell size was effective in improving
vibration extraction accuracy.
The gain parameters that adjust the convergence speed of the BMFLC weights
were found to be $2^{-7}$, $2^{-8}$, and $2^{-9}$ appropriate for the current evaluation.

In the vibration experiments using a antiphase vibratory apparatus,
it was found that an unintended frequency component was included in the excitation vibration
of the antiphase vibratory apparatus.

In the Tremor suppression experiment using a tremor reproduction device,
a decrease in the frequency component of the excitation vibration was observed,
but an increase in the vibration component of other frequency components
was observed due to the influence of unintended frequency components and other factors.


Future work includes clarifying the causes of unintended frequency components
in antiphase vibratory apparatus, evaluating vibration control in unsteady vibration,
and studying algorithms that can estimate vibration displacements with high precision.


\section{Ackhowledgment}\label{section:ackhowledgment}
Acknowledgments are omitted for double-blind review.
